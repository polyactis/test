\documentclass[a4paper,10pt]{article}
\usepackage[dvips]{color,graphicx}
\usepackage[dvips, bookmarks, colorlinks=false]{hyperref}

%opening
\title{Math650 Homework 3.1}
\author{Yu Huang}

\begin{document}

\maketitle

\begin{abstract}
Fish Oil and Blood Pressure
\end{abstract}

\section{Introduction}
The question is whether fish oil could reduce the blood pressure.

\section{Materials and Methods}
Questions from \emph{Statistical Sleuth}, Chap 1, No.12, Chap 2, No.13, No.14.

Data from the CDROM. Using software R.

\section{Results}

\begin{tabular}{|r|r|}
\hline
Fish Oil Average&6.571429\\
Regular Oil Average&-1.142857\\
Fish Oil standard deviation&5.8554\\
Regular Oil standard deviation&3.184785\\
pooled estimate of standard deviation&4.713203\\
standard error for the difference&2.519313\\
degree of pooled estimate of standard deviation&12\\
corresponding t-dist 97.5th percentile&2.179\\
95\% confidence interval of $\mu_2 - \mu_1$ & 2.224702-13.20387\\
t-statistic for testing equality & 3.062059\\
one-sided p-value(looking up table) & 0.005\\
\hline
below is generated by R&\\
\hline
corresponding t-dist 97.5th percentile&2.178813\\
95\% confidence interval of $\mu_2 - \mu_1$ & 2.225174-13.20340\\
one-sided p-value & 0.00493062\\
\hline
\end{tabular}

All R codes are appended(\ref{appendix}).

\section{Conclusion and Discussion}
Regarding question Chap1 No. 12, though the volunteers do not constitute a random sample from any population and we can't do \emph{Inference to Population}, the \emph{Allocation of Units to Groups} is random. It's still valid to do \emph{Causal Inference}.

Based on the 0.005 p-value, it's very unlikely that the reduction of blood pressure after taking fish oil happens by chance.

\section{Appendix}
\label{appendix}
\begin{verbatim}
t_test_func = function(sample_data1, sample_data2)
{
	len_f = length(sample_data1)
	len_r = length(sample_data2)
	
	mean_f = mean(sample_data1)
	cat("mean_f:", mean_f, "\n")
	mean_r = mean(sample_data2)
	cat("mean_r:", mean_r, "\n")
	cat("mean difference:", mean_f-mean_r, "\n")
	
	sd_f = sd(sample_data1)
	cat("sd_f:", sd_f, "\n")
	sd_r = sd(sample_data2)
	cat("sd_r:", sd_r, "\n")
	
	degrees_of_freedom_of_pooled_sd = len_f + len_r -2
	cat("degrees_of_freedom_of_pooled_sd:", degrees_of_freedom_of_pooled_sd, "\n")
	
	pooled_sd = sqrt( ( (len_f-1)*sd_f^2 + (len_r-1)*sd_r^2  )/degrees_of_freedom_of_pooled_sd )
	cat("pooled_sd:", pooled_sd, "\n")
	
	standard_error_for_the_difference = pooled_sd * sqrt(1/len_f + 1/len_r)
	cat("standard_error_for_the_difference:", standard_error_for_the_difference, "\n")
	
	#by looking up a table with 12=degrees_of_freedom_of_pooled_sd (only for 1st part of hw3)
	percentile_97_5th = 2.179
	cat("percentile_97_5th, df=12:", percentile_97_5th, "\n")
	conf_interv_of_difference_of_mu_lower = mean_f-mean_r- percentile_97_5th*standard_error_for_the_difference
	conf_interv_of_difference_of_mu_upper = mean_f-mean_r + percentile_97_5th*standard_error_for_the_difference
	cat("conf_interv_of_difference_of_mu_lower, df=12:", conf_interv_of_difference_of_mu_lower, "\n")
	cat("conf_interv_of_difference_of_mu_upper, df=12:", conf_interv_of_difference_of_mu_upper, "\n")
	
	t_stat = (mean_f - mean_r - 0)/standard_error_for_the_difference
	cat("t_stat:", t_stat, "\n")
	
	#by looking up a table with degrees_of_freedom_of_pooled_sd=12
	#p_value = 1-0.995 = 0.005 corresponding to t_stat=3.055, actual p_value should be less than this value
	
	
	#below is getting confidence interval and p_value through R
	
	percentile_97_5th = qt(0.025, degrees_of_freedom_of_pooled_sd, lower.tail=FALSE)
	cat("percentile_97_5th:", percentile_97_5th, "\n")
	
	conf_interv_of_difference_of_mu_lower = mean_f-mean_r- percentile_97_5th*standard_error_for_the_difference
	conf_interv_of_difference_of_mu_upper = mean_f-mean_r + percentile_97_5th*standard_error_for_the_difference
	cat("conf_interv_of_difference_of_mu_lower:", conf_interv_of_difference_of_mu_lower, "\n")
	cat("conf_interv_of_difference_of_mu_upper:", conf_interv_of_difference_of_mu_upper, "\n")
	p_value = pt(t_stat, degrees_of_freedom_of_pooled_sd, lower.tail=FALSE)
	cat("p-value of t_stat", p_value, "\n")

}

#chap2 No13, No14

data = read.csv("/usr/local/doc/statistical_sleuth/ASCII/ex0112.csv")
sample_data1 = data[data$DIET=="fishoil",]$BP
sample_data2 = data[data$DIET=="regularoil",]$BP
t_test_func(sample_data1, sample_data2)
\end{verbatim}

Output is this:
\begin{verbatim}
mean_f: 6.571429
mean_r: -1.142857
mean difference: 7.714286
sd_f: 5.8554
sd_r: 3.184785
degrees_of_freedom_of_pooled_sd: 12
pooled_sd: 4.713203
standard_error_for_the_difference: 2.519313
percentile_97_5th, df=12: 2.179
conf_interv_of_difference_of_mu_lower, df=12: 2.224702
conf_interv_of_difference_of_mu_upper, df=12: 13.20387
t_stat: 3.062059
percentile_97_5th: 2.178813
conf_interv_of_difference_of_mu_lower: 2.225174
conf_interv_of_difference_of_mu_upper: 13.20340
p-value of t_stat 0.00493062
\end{verbatim}
\end{document}
