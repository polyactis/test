\documentclass[a4paper,10pt]{article}
\usepackage{makeidx}
\usepackage[dvips]{color,graphicx}
\usepackage[dvips, bookmarks, colorlinks=false]{hyperref}
\makeindex

%opening
\title{Problems of Monte Carlo Strategies in Scientific Computing}
\author{Yu Huang}

\begin{document}

\maketitle

\begin{abstract}
This is a collection of solutions and notes of the book written by Jun S. Liu.
\end{abstract}
\tableofcontents

\section{problems of Chapter 2}
\subsection{problem 2.1}

\begin{math}
\int_{0}^{1} sin^2(1/x)dx = \int_{0}^{1}1/2(cos(0)-cos(2/x))dx = \int_{0}^{1}1/2(1-cos(2/x))dx = 1/2 - 1/2\int_{0}^{1}cos(2/x)dx
\end{math}

I don't know how to solve $\int_{0}^{1}cos(2/x)dx$. Several results returned by google search are not satisfactory. Instead, i use mathematica(matlab is down for some reason on hpc-cmb), $NIntegrate[Sin[1/x]*Sin[1/x], \{x, 0, 1\}]$ to get 0.673427 and regard this as the deterministic answer though mathematica fails to converge.

SimulateIntegral.py uses the plain Monte carlo method. $1e^4$ samplings gives mean: 0.673037731012, and std: 0.00282346219155 and $1e^7$ samplings outputs a much more accurate answer with a smaller std: 0.673477836344 (7.40591084413e-05) .

\subsection{problem 2.2}
We want to generate random variables that follow a distribution, $\pi(x)$. Suppose $l(x)=c\pi(x)$ is computable with $c$ unknown. We known how to sample from a trial distribution $g(x)$. There exists "covering constant", $M$ so that the envelope property [i.e. $Mg(x)>=l(x)$] is satisfied for all x.

\begin{enumerate}
\item draw a sample from $g()$, and compute the ratio $r=l(x)/Mg(x)$
\item flip a coin with probability $r$.
\end{enumerate}


\subsection{problem 2.3}
The only trouble is how to get the analytic form of $t_v(x; \mu, \Sigma)$ to carry out the differentiation.

\section{An EM example}
The problem is that there's a sequence of bits. Each bit comes from one of two benoulli distributions, say $p_1$ and $p_2$. There's a probability $r$ for the bit to come from benoulli $p_1$ and $1-r$ from $p_2$. Given a sequence of bits, infer $p_1$, $p_2$ and $r$.

Using missing data approach,

\end{document}
